% !TEX root = pa_doc.tex
\begin{markdown}

# Development

## Agile Approach

For development of the ZHAWo prototype we used an iterative agile approach. The code base was managed on a single GitHub repository \cite{OurGithub} for both front- and backend. User stories were tracked using GitHub issues and the sprints were organized using GitHub project boards. Over the course of the project, we structured the development into six sprints of 2 weeks. We regularly considered feedback of other students and our own review of our practices and used technologies in our planning. The flexibility of using only JavaScript and PWA technologies enabled us to rapidly prototype new ideas without the hassle of having to maintain two different code bases.
User stories were implemented and tested in feature branches and merged into the master branch as part of the sprint reviews. This practice ensured that we had - for the most part - stable new iterations of the application to deploy for user feedback.

## Continuous Integration \& Deployment

For continuous integration we used Travis CI \cite{Travis} in combination with Codecov \cite{Codecov} to track test coverage. With good integration of both these tools into GitHub - for example reporting of build status and test coverage changes in GitHub comments on each pull request - we achieved good code and test quality across feature implemenations and sprints.
While the goal is to eventually host our application on a ZHAW server, in the scope of this project the prototype was deployed to Heroku \cite{Heroku} and shared amongst students for quick feedback on new features.

\newpage

## Product Backlog

The following planned user stories have been implementat in the scope of this project. We put the focus on implementing the schedule context of ZHAWo in detail and provide prototypical functionality for the other primary functionalites. For implementation details please refer to section 3.

\begin{itemize}
  \item \textbf{US01}: As a user I want to save my credentials/username
  \item \textbf{US02}: As a user I want the app to work even when I don't have network connection
  \item \textbf{US03}: As a user I want to switch between contexts (schedule, menus, room search, vszhaw)
  \item \textbf{US10}: As a user I want to view my timetable/schedule for a day
  \item \textbf{US11}: As a user I want to vew my timetable for a week
  \item \textbf{US12}: As a user I want to navigate to the current day
  \item \textbf{US13}: As a user I want to navigate between days when using the day view (timetable)
  \item \textbf{US14}: As a user I want to navigate between weeks (timetable)
  \item \textbf{US15}: As a user I want to navigate to a specific date in a month view (timetable)
  \item \textbf{US16}: As a user I want to view a specific rooms timetable
  \item \textbf{US17}: As a user I want to view a specific classes timetable
  \item \textbf{US18}: As a user I want to view a specific courses timetable
  \item \textbf{US19}: As a user I want to view a specific persons timetable
  \item \textbf{US20}: As a user I want to have a detailed view of my events
  \item \textbf{US30}: As a user I want to find currently unoccupied rooms
  \item \textbf{US56}: As a user I want to see prices for all menus
  \item \textbf{US58}: As a user I want to view a specific mensas menu plan
  \item \textbf{US70}: As a user I want to view vszhaw blog posts/event announcements
\end{itemize}

\newpage

The following user stories are planned to be implemented in the finished application. This list will of course be extended and adjusted based on user feedback and our own review.

\begin{itemize}
  \item \textbf{US31}: As a user I want an overview of my campus with highlighted buildings where there are unoccupied rooms
  \item \textbf{US32}: As a user I want a floor plan of each floor per building with highlighted unoccupied rooms
  \item \textbf{US33}: As a user I want to navigate between buildings through the overview of my campus
  \item \textbf{US34}: As a user I want to navigate between floor plans of a building
  \item \textbf{US35}: As a user I want to see until when a room is unoccupied
  \item \textbf{US36}: As a user I want to filter my search to only show rooms that are unoccupied for at least x hours/minutes
  \item \textbf{US50}: As a user I want to view the mensa menu for my campus for a day
  \item \textbf{US51}: As a user I want to view the mensa menu for my campus for a week
  \item \textbf{US52}: As a user I want to navigate to the mensa menu of the current day
  \item \textbf{US53}: As a user I want to navigate between days when using the mensa menu day view
  \item \textbf{US54}: As a user I want to navigate between weeks when using the mensa menu week view
  \item \textbf{US55}: As a user I want to navigate to a specific date in a month view (mensa menu)
  \item \textbf{US57}: As a user I want to navigate between menus of different days
\end{itemize}

\end{markdown}
