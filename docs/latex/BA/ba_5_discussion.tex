% !TEX root = ba_doc.tex
\begin{markdown}
\section{Discussion} \label{discussion}

Using a full JavaScript technology stack for front- and backend and Progressive Web Apps has proven to work well with an agile development approach. While developing native applications, having to maintain different code bases can be awkward in combination with agile principles. The strength of the agile development approach that we have chosen is that we can prototype new features rapidly and can receive user feedback from both Android and iOS users immediately. In addition to that, not having to stay up to date with best practices, libraries and frameworks for different languages such as Java/Kotlin and Swift/Objective C for Android and iOS development respectively allowed us to implement new features at a faster pace.

While there are obvious advantages to building ZHAWo as a cross platform PWA, we have also identified some weaknesses. Installing an application not through the App Store or the Google Play Store is unexpected for most users. There are also some inconsistencies in support offered by different platforms. We attribute most of our issues to PWA technologies being relatively new and expect both support and users familiarity with the concept of PWAs to increase in the near future. 

We identified additional issues inconsistencies in rendering of our application on different smartphone browsers. Especially the scaling of the SVG images of the building and floor plans was not optimal on some smaller phones and specifically on Huawei smartphones. These issues are not specific to PWA technology, but arise for all mobile websites and should be considered when deciding between using native technologies or a PWA.

The user survey confirmed that a room search feature is in high demand amongst students of the ZHAW and our visual implementation with navigation between building and floor plans was well received. With the updates to the official CampusInfo application for iOS, a good solution to view timetable and mensa menus was less important than at the start of this project, however to us it makes sense to combine both these feature and the room search functionality in one single application.

\end{markdown}

