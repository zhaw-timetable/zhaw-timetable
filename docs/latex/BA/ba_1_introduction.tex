% !TEX root = ba_doc.tex
\begin{markdown}
\section{Introduction} \label{introduction}

Students of the Zurich University of Applied Sciences (ZHAW) have to visit multiple websites or use different applications on a daily basis in order to get information about their schedule, the offered menus in their campus mensa and events organized by the Verein Studierende ZHAW (vszhaw).

For their schedule, a student can either visit the official site \cite{Stundenplan} or use the official CampusInfo application for either Android \cite{AppAndroid} or iOS \cite{AppIOS}. The official site was designed for use on desktop browsers and is not optimised for responsiveness and display on phones. And while the official Android application is well maintained and offers a good user experience and a lot of additional features - such as direct access to public transport timetables and mensa menu plans - its iOS counterpart was seemingly lagging behind in development and at least at the start of this project did not offer the same user experience. The most glaring issue with the iOS application was the lack of offline functionality. When a user's network cut out, even the schedule information that was previously loaded can no longer be accessed after navigating away. The iOS application has since received an update with a much improved design and offline functionality. This difference in quality and features is a common occurrence because the development of native applications requires two separate code bases for Android and iOS. 

When students want to check the mensa menus of their campus, they have the option of visiting the SV groups site \cite{SVSite} or to use the official Android or iOS application. These options suffered from the same issues as previously explained for the schedule and were also improved in a recent update.

With ZHAWo, our goal is to provide students with an improved application to have access to their schedule and mensa menus in one single cross-platform application. Additionally, with ZHAWo students can look for unoccupied rooms. This functionality was - until about two years ago - provided by an Android application that was no longer maintained and eventually disappeared from the Google Play Store. While the official study rooms at, for example, the Technikum campus offer a space for both quiet work as well as group projects, in our experience as students it was very convenient to have a service to quickly look up free rooms without having to walk from room to room. Another feature ZHAWo provides is the integration of news \cite{VszhawNews} and events \cite{VszhawCalendar} of the vszhaw directly into the application. We aim to reduce the effort that is needed to stay up to date with the vszhaw and hope that both students and the vszhaw can profit from this.

We established a prototype for ZHAWo, a Progressive Web Application (PWA) for students of the Zurich University of Applied Sciences (ZHAW) in a previous project. The goal of this project is to transform the prototype to a production-ready application that can be distributed to and used by the students. The focus in this work is put on development of a feature-rich, production-ready PWA while evaluating advantages and disadvantages of using Progressive Web App technologies as well as user reception of our application.

By using Progressive Web Application (PWA) technologies \cite{PWA} in combination with JavaScript frameworks for both front- and backend, we achieve a consistent user experience on desktop, Android and iOS devices. We eliminate the issue of having to maintain separate code bases for different platforms while still being able to provide a native feel and functionality. PWA features such as offline caching of HTTP requests allow us to overcome previously mentioned issues with reliability on spotty networks.

An additional advantage we gain by using PWA technologies and the same programming language across the full stack is fast prototyping in an agile development process. Development of additional features and functionality can be achieved at a much faster rate with a single code base across all platforms.

\newpage

\end{markdown}
