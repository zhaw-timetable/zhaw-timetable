% !TEX root = ba_doc.tex
\begin{markdown}

# Discussion

We were able to implement a prototype of ZHAWo as a Progressive Web App which covers all primary functions that were planned. Students can access their schedule, mensa plans and the news and event feed from vszhaw. Additionally, they can get a list of free rooms that can then be used as a distraction free work space. We were able to share the prototype with a small group of students and user feedback for our application was positive.

Using a full JavaScript technology stack for front- and backend and Progressive Web Apps has proven to work well with an agile development approach. While developing native applications, having to maintain different code bases can be awkward in combination with agile principles. The strength of the agile development approach that we have chosen is that we can prototype new features rapidly and can receive user feedback from both Android and iOS users immediately. In addition to that, not having to stay up to date with best practices, libraries and frameworks for different languages such as Java/Kotlin and Swift/Objective C for Android and iOS development respectively allowed us to implement new features at a faster pace.

While there are obvious advantages to building ZHAWo as a cross platform PWA, we have also identified some weaknesses. Installing an application not through the App Store or the Google Play Store is unexpected for most users. There are also some inconsistencies in support offered by different platforms. We attribute most of our issues to PWA technologies being relatively new and expect both support and users familiarity with the concept of PWAs to increase in the near future.

In a second part of this project, we aim to improve the user experience further and extend ZHAWo with more features.

\end{markdown}
