% !TEX root = ba_doc.tex
\begin{markdown}
\section{Development} \label{development}

## Agile Approach

For development of ZHAWo we used an iterative agile approach. The code base was managed on a single GitHub repository \cite{OurGithub} for both front- and backend. User stories were tracked using GitHub issues and the sprints were organised using GitHub project boards. We structured the development into sprints of 2 weeks. We regularly considered feedback of other students and our own review of our practices and used technologies in our planning. The flexibility of using only JavaScript and PWA technologies enabled us to rapidly prototype new ideas without the hassle of having to maintain two different code bases.
User stories were implemented and tested in feature branches and merged into the master branch as part of the sprint reviews. This practice ensured that we had stable iterations of the application to deploy for user feedback.

## Continuous Integration \& Deployment

For continuous integration we used Travis CI \cite{Travis} in combination with Codecov \cite{Codecov} to track test coverage. With good integration of both these tools into GitHub - for example reporting of build status and test coverage changes in GitHub comments on each pull request - we achieved good code and test quality across feature implementations and sprints.
Each new iteration was deployed to a ZHAW server and was directly pushed to our users on their next visit, without having to install an update through an app store as it would be the case with native applications.

## Primary Functions

To structure the development and implementation, we established four primary functions that our application should provide.

\begin{itemize}
  \item \textbf{Timetable}: A user can access their schedule and look up schedules of lecturers, classes, courses and rooms.
  \item \textbf{Menu plans}: A user can access menu plans of the different campus mensas across the ZHAW.
  \item \textbf{Room search}: A user can look for and find free rooms for a specific time-frame and location.
  \item \textbf{Student events}: A user can access vszhaw news and events to bring more attention to student parties and events.
\end{itemize}

## Product Backlog

We divided each primary function into seperate smaller user stories to plan our sprints. Additionaly we have defined general user stories that are not directly related to any of the primary functions. All user stories are listed in the following sections.

### General

\begin{itemize}
  \item \textbf{US01}: As a user I want to save my credentials/username
  \item \textbf{US02}: As a user I want the app to work even when I don't have network connection
  \item \textbf{US03}: As a user I want to switch between contexts (schedule, menus, room search, vszhaw)
\end{itemize}

### Timetable

\begin{itemize}
  \item \textbf{US10}: As a user I want to view my timetable/schedule for a day
  \item \textbf{US11}: As a user I want to view my timetable for a week
  \item \textbf{US12}: As a user I want to navigate to the current day
  \item \textbf{US13}: As a user I want to navigate between days when using the day view (timetable)
  \item \textbf{US14}: As a user I want to navigate between weeks (timetable)
  \item \textbf{US15}: As a user I want to navigate to a specific date in a month view (timetable)
  \item \textbf{US16}: As a user I want to view a specific room's timetable
  \item \textbf{US17}: As a user I want to view a specific class's timetable
  \item \textbf{US18}: As a user I want to view a specific course's timetable
  \item \textbf{US19}: As a user I want to view a specific person's timetable
  \item \textbf{US20}: As a user I want to have a detailed view of my events
\end{itemize}

### Menu plans

\begin{itemize}
  \item \textbf{US50}: As a user I want to view the mensa menu for my campus for a day
  \item \textbf{US51}: As a user I want to view the mensa menu for my campus for a week
  \item \textbf{US52}: As a user I want to navigate to the mensa menu of the current day
  \item \textbf{US53}: As a user I want to navigate between days when using the mensa menu day view
  \item \textbf{US54}: As a user I want to navigate between weeks when using the mensa menu week view
  \item \textbf{US55}: As a user I want to navigate to a specific date in a month view (mensa menu)
  \item \textbf{US56}: As a user I want to see prices for all menus
  \item \textbf{US57}: As a user I want to navigate between menus of different days
  \item \textbf{US58}: As a user I want to view a specific mensa menu plan
\end{itemize}

### Room search

\begin{itemize}
  \item \textbf{US30}: As a user I want to find currently unoccupied rooms
  \item \textbf{US31}: As a user I want an overview of my campus with highlighted buildings where there are unoccupied rooms
  \item \textbf{US32}: As a user I want a floor plan of each floor per building with highlighted unoccupied rooms
  \item \textbf{US33}: As a user I want to navigate between buildings through the overview of my campus
  \item \textbf{US34}: As a user I want to navigate between floor plans of a building
  \item \textbf{US35}: As a user I want to see until when a room is unoccupied
  \item \textbf{US36}: As a user I want to filter my search to only show rooms that are unoccupied for at least x hours/minutes
\end{itemize}

### Student events

\begin{itemize}
  \item \textbf{US70}: As a user I want to view vszhaw blog posts/event announcements
  \item \textbf{US71}: As a user I want to see upcoming vszhaw events (f. ex. next party)
\end{itemize}

\end{markdown}
