% !TEX root = pa_doc.tex
\begin{markdown}

# Introduction

Students of the Zürcher Hochschule für Angewandte Wissenschaften (ZHAW) vist mutliple websites daily in order to get the information they need.

\begin{itemize}
  \item \textbf{School Scedule}: http://stundenplan.zhaw.ch/
  \item \textbf{Menu plan}: http://technikum.sv-restaurant.ch/de/menuplan/: Menu plan
  \item \textbf{School events}: https://www.vszhaw.ch/events/
  \item \textbf{Find free Rooms}: Go from room to room
\end{itemize}

% TODO change this sentence below -> less complicated
% bitz en ahgriff nid?

However, apart from the official Android application \cite{DUMMY}, they are either not well suited for use on mobile devices in the case of the official schedule site \cite{DUMMY} or feel outdated.

% TODO write something about issue of maintaining code for both Android and Iphone


By using progressive web application (PWA) technologies \cite{OurReadme} we were able to build a Cross-Platform application.
We want to provide our Users a fast, Reliable and Engaging experience across all Platforms. That is why we have chosen to develop zhawo as a Progressive Web App. Progressive Web Apps are web apps that behave like native apps. That means the application can be accessed by any device that has a browser, whilst still gives the user the look and feel of a native app.

<<<<<<< HEAD
## Goals & Primary Functions
=======


## Goals + Primary Functions
>>>>>>> b0997382d9c860904ff601469f3ebc1d9526fd72
%TODO how to make &

\begin{itemize}
  \item \textbf{Timetable}: students often need to quickly check what courses they have next and where they need to go. The availability of this feature is very important and has been noted as being an issue with the "official" timetable app. Using intelligent caching and a modern user interface, we aim to improve this aspect.
  \item \textbf{Room search}: when we first started there was an application, that allowed students to search for free rooms all over their campus. This was very useful as study programmes at the ZHAW put a huge focus on group projects, but with limited quiet public working spaces, finding a free room was often vital. The app that only ever existed for Android devices has since completely disappeared.
  \item \textbf{Menu plans}: There are menu plans on the official provider of the mensa food's website. However, this site is rather unknown and we plan to include an easy way to check today's and upcoming menus and prices.
  \item \textbf{Student events}: We plan to work with vszhaw to bring more attention to student parties and events, by integrating their event feed into zhawo.
\end{itemize}

\end{markdown}
